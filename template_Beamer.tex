\documentclass[12pt]{beamer} % Changed to 24pt for normal text
\usetheme{default} % Using the default theme as a base

% Load xcolor for defining custom colors
\usepackage{xcolor}
\usepackage{graphicx}
\usepackage[T1]{fontenc}


% Define a gold color
\definecolor{mygold}{RGB}{190,157,91} % A common shade of gold
\definecolor{bgblack}{RGB}{47,47,47}
\definecolor{mywhite}{RGB}{245,245,245}
\definecolor{mygray}{RGB}{219, 214,202}
\definecolor{myred}{RGB}{161, 57, 49}

% --- Global Color Settings ---
% Set the overall background of the slide canvas to black
\setbeamercolor{background canvas}{bg=bgblack}

% Set normal text to be mywhite on a bgblack background
\setbeamercolor{normal text}{fg=mywhite,bg=bgblack} % Changed fg to mywhite

\setbeamercolor{frametitle}{fg=mywhite}
\setbeamercolor{title}{fg=mywhite} 

\setbeamercolor{caption name}{fg=mywhite}
\setbeamercolor{itemize item}{fg=mygray}
\setbeamercolor{enumerate item}{fg=mygray}

\setbeamercolor{bibliography item}{fg=mywhite}          % 例如文献编号 [1]
\setbeamercolor{bibliography entry author}{fg=mywhite}  % 作者名
\setbeamercolor{bibliography entry title}{fg=mywhite}   % 文献标题 (您提到已是白色,为保险起见设置)
\setbeamercolor{bibliography entry location}{fg=mywhite} % 发表位置 (如期刊名、出版社)
\setbeamercolor{bibliography entry note}{fg=mywhite}    % 其他注释 (如年份、卷号、页码、DOI、URL文本)

% --- Font Size Settings ---
% Set frame title font size to 36pt (baselineskip 43pt)
\setbeamerfont{frametitle}{size=\fontsize{21}{12}\selectfont}


% --- Customize the Frametitle Template ---
\setbeamertemplate{frametitle}{%
	\ifx\insertframetitle\@empty\else 
	\nointerlineskip% Prevents some unwanted vertical skips
	\vspace*{0.15\textheight} % This might need tweaking due to larger title font
	
	\begin{center} % Center the title and the line
		% Display the frame title using the defined font and color
		{\usebeamerfont{frametitle}\usebeamercolor[fg]{frametitle}\insertframetitle}%
		%\par % Ensure a paragraph break after the title
		%\vspace{1ex} % Add a small vertical space between the title and the line
		
		% Draw the golden line
		% Adjust \textwidth multiplier for line width (e.g., 0.8\textwidth for 80% of text width)
		% Adjust the '1.2pt' for line thickness
		{\color{mygold}\rule{1.0\textwidth}{1.5pt}}%
	\end{center}
	
	% Add some space after the title/line block, before the main frame content begins
	%\vspace{1ex}
	\fi%
}

% --- Page Numbering Settings ---
% Set the color for the page number (if not already set or to override)
\setbeamercolor{page number in head/foot}{fg=mywhite}

% Customize the footline to display page numbers starting from 1 on the second slide
\setbeamertemplate{footline}{%
  \ifnum\value{framenumber}>1% Only show after the first frame
    \hfill% Push page number to the right
    \usebeamercolor[fg]{page number in head/foot}%
    \usebeamerfont{page number in head/foot}%
    \the\numexpr\value{framenumber}-1\relax% Display (current frame number - 1)
    \hspace*{2ex}% Add some space after the page number
    \vspace{2ex}
  \fi%
}

% Optional: If you want to remove the default navigation symbols (like arrows, etc.)
%\setbeamertemplate{navigation symbols}{}


%\setbeameroption{hide notes} % Only slides
%\setbeameroption{show only notes} % Only notes
\setbeameroption{show notes on second screen=right} % Both




\usepackage{cite}
% Removes icon in bibliography
\setbeamertemplate{bibliography item}[text]



% --- Presentation Metadata ---
\title{Beamer Template with Custom Styling}
\author{TeXstudio Team (Enhanced)}
\date{} % It's good practice to include the date

\begin{document}
	
	\begin{frame}[plain]
		\maketitle 
	\end{frame}
	
	% --- Content Frame 1 ---
	\begin{frame}
		\frametitle{First Content Frame Title}
        \note[item]{\large{Welcome to the talk!}}
        \note[item]{As you can see, this slidedeck is a work in progress.}
		This is some sample text on the black background.
		\begin{itemize}
			\item Item 1 will be in white.
			\item Item 2 as well.
		\end{itemize}
		The golden line should appear below the title.
	\end{frame}
	
	% --- Content Frame 2 ---
	\begin{frame}
		\frametitle{Another Frame with a Longer Title}
		More content goes here.
		\begin{enumerate}
			\item Numbered item A.
			\item Numbered item B.
		\end{enumerate}
		The title positioning and line should be consistent.
	\end{frame}
	
	% --- Frame with Red Keyword Highlighting ---
	\begin{frame}
		\frametitle{Highlighting Key Information} % You can change this title
		
		On this slide, we will demonstrate how to make certain words stand out.
		You can use red characters for emphasis on specific terms like important.
		
		For a stronger visual cue, you can place keywords within a \colorbox{myred}{red block}.
		Notice that the text color inside the red block will be your default text color (`mywhite`), which provides good contrast. For example, this is a \colorbox{myred}{critical point}.
	\end{frame}
	
	
	% --- Frame without a title (to test the conditional in the template) ---
	\begin{frame}
		This frame intentionally has no \texttt{\textbackslash frametitle}.
		Therefore, no title or golden line should appear at the top.
		Only the main content with white text on a black background\cite{shamir2010learning}.
	\end{frame}
	
	% --- Frame with Figure on Left and Text on Right ---
	\begin{frame}
		\frametitle{Figure and Text Side-by-Side} % You can change this title
		
		% The columns environment will be placed below the title and the gold line
		\begin{columns}[T] % The [T] option aligns the top of the columns
			
			% Left Column for the Figure
			\begin{column}{0.50\textwidth} % Adjust width as needed (e.g., 0.45\textwidth)
				\centering % Optional: centers the figure in the column
				% If you want a caption for your figure, you can use the figure environment:
				\begin{figure}
				    \includegraphics[width=0.8\linewidth]{dikaer.jpg}
					%\caption{This is a caption for the figure.}
				\end{figure}
			\end{column}
			
			% Right Column for the Text
			\begin{column}{0.50\textwidth} % Adjust width as needed (e.g., 0.45\textwidth)
				This is where your descriptive text will go.
				You can use standard LaTeX formatting here.
				\begin{itemize}
					\item Explain key aspects of the figure.
					\item Provide supporting details.
					\item Or list important takeaways.
				\end{itemize}
				More text can follow here, describing the content further.
			\end{column}
			
		\end{columns}
	\end{frame}

    \begin{frame}[allowframebreaks]{References}
    	\bibliographystyle{plain}
    	%\bibliography{mybeamer} also works
    	\bibliography{./template_Beamer.bib}
    \end{frame}
	
\end{document}
